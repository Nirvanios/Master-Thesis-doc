% Tento soubor nahraďte vlastním souborem s obsahem práce.
%=========================================================================
% Autoři: Michal Bidlo, Bohuslav Křena, Jaroslav Dytrych, Petr Veigend a Adam Herout 2019
\chapter{Úvod}

Tento text slouží jako ukázkový obsah šablony a současně rekapituluje nejdůležitější informace z předpisů a poskytuje další užitečné informace, které budete potřebovat pro tvorbu technické zprávy ke svojí práci. Než se šablonou budete dále pracovat, je třeba vědět, jak ji správně použít. To je stručně uvedeno v~příloze \ref{jak}.

I když některým studentům pro napsání dobré diplomové práce (bakalářská práce je také diplomová -- dostává se za ni diplom) stačí znát a dodržovat oficiální formální požadavky uvedené ve směrnicích a typografické zásady, často je výhodné před započetím psaní zjistit, jaké jsou osvědčené postupy pro psaní odborného textu a jak si práci usnadnit. Někteří vedoucí svým studentům připravili popisy osvědčených postupů, které vedly k desítkám úspěšně obhájených prací. Výběr nejzajímavějších postupů, které měli autoři této šablony k~dispozici ve chvíli její tvorby, je v níže uvedených kapitolách. Má-li Váš vedoucí svoji stránku s doporučenými postupy, tyto kapitoly můžete vynechat a řídit se pokyny svého vedoucího. Pokud takovou stránku nemá, může být přečtení níže uvedeného textu vhodnou přípravou na konzultaci o plánované struktuře a náplni textu práce.

Diplomová práce je rozsáhlé dílo a tomu odpovídá i technická zpráva. Ne každý je schopen si sednout a jednoduše ji napsat. Je třeba vědět, kde začít a jak postupovat. Jedním z možných přístupů je začít psaním klíčových slov a abstraktu, abyste si ujasnili, co je v~práci nejdůležitější. O tom pojednává kapitola \ref{abstrakt}.

Po sepsání abstraktu se lze pustit do psaní samotného textu technické zprávy. Typicky si nejprve připravíme základní strukturu práce, kterou pak budeme plnit textem. Kapitola \ref{struktura} se zabývá základními informacemi a radami pro psaní odborného textu, které Vám pomohou vyhnout se začátečnickým chybám, a stanovením nadpisů kapitol a přibližných rozsahů jednotlivých částí práce. V závěru kapitoly je pak uveden přístup, kterým si lze psaní technické zprávy značně usnadnit.

Diplomové práce v oblasti informačních technologií mají určitou typickou strukturu. Po~úvodu bude následovat kapitola či kapitoly zabývající se shrnutím současného stavu, který bude v následujících kapitolách zhodnocen a bude navrženo řešení, které bude implementováno a otestováno. V závěru pak budou výsledky vyhodnoceny a bude navržen budoucí vývoj. I když se názvy a rozsahy kapitol v různých pracích liší, vždy tam lze najít kapitoly odpovídající této struktuře. Kapitola \ref{kapitoly} se zabývá obsahy typických kapitol, které se v diplomových pracích z oblasti IT vyskytují. Většina studentů ve svojí práci pravděpodobně využije pouze určitou podmnožinu popsaných kapitol, která je pro jejich práci relevantní. Uvedené popisy a rady mohou pomoci jak s~rozhodnutím, zda danou kapitolu uvést, tak i~s~vnitřní strukturou a samotným obsahem kapitoly.

Za závěrečnou kapitolou práce vždy následuje seznam použité literatury. Citacemi, které tento seznam tvoří, a odkazy na ně se zabývá kapitola \ref{citace}. Byť to tak nezkušený student nemusí vnímat, je seznam použité literatury a odkazy na něj pro práci zcela zásadní. Hodnocení práce s literaturou a citací tvoří jednu z důležitých částí posudku oponenta a bude-li chybět jediná položka, může to vést k hodnocení stupněm F, následnému disciplinárnímu řízení za plagiátorství a k vyloučení z nedokončeného studia. Nesprávná práce se zdroji může mít i další důsledky -- v roce 2018 stála křesla dva členy české vlády. Proto prosím citacím věnujte odpovídající pozornost. 

Po dokončení textu je nutné zjistit, jaké požadavky jsou kladeny na vysokoškolskou kvalifikační práci na FIT VUT v~Brně, a dořešit případné nedostatky. Formální požadavky jsou uvedeny ve směrnicích a na webových stránkách, které jsou zmíněny v kapitole \ref{formality}. Tato kapitola obsahuje i požadované rozsahy jednotlivých typů prací a další vybrané informace z~předpisů a doporučení. V závěru kapitoly je uveden přehled nejčastějších chyb, se kterými se oponenti setkávají a kterým byste se měli vyhnout. Hodnocení formální úpravy práce je pak další z důležitých součástí posudku oponenta.

Po odstranění formálních nedostatků lze práci odevzdat. Před odevzdáním práce si můžete projít kontrolní seznam (tzv. \uv{checklist}) uvedený v příloze \ref{checklist}. Samotné odevzdání listinné i elektronické verze práce je pak popsáno v kapitole \ref{odevzdani}.

V závěrečné kapitole \ref{zaver} je pak uvedeno shrnutí toho, co se lze přečtením tohoto textu dozvědět, a to nejdůležitější, na co je třeba myslet před odevzdáním práce.


\chapter{Teorie}
\label{teorie}

\chapter{Návrh řešení}
\label{navrh resení}

\chapter{Implementace}
\label{implementace}

\chapter{Testování}
\label{testovani}

\chapter{Závěr}
\label{zaver}

V tomto textu bylo uvedeno, jak začít s tvorbou bakalářské či diplomové práce, napsat abstrakt, připravit základní strukturu práce a co uvést do jednotlivých kapitol. Při tom bylo vysvětleno, že bakalářská práce je také diplomová a je třeba k ní přistupovat stejně zodpovědným způsobem. Následně byla věnována pozornost bibliografickým citacím a formální stránce práce. V předposlední kapitole jsou uvedeny důležité informace k odevzdání v listinné i v elektronické podobě.

Je třeba zdůraznit, že diplomová práce je unikátním individuálním dílem, které vzniká pod vedením zkušeného odborníka. Ať už je v této šabloně uvedeno cokoliv, závazné jsou pouze oficiální pokyny na stránkách fakulty. Pro konkrétní diplomovou práci je potřeba vždy zvažovat, co je z výše uvedeného textu relevantní a co nikoliv a řídit se především pokyny vedoucího, který rozumí dané problematice a je tak schopen poskytnout ty nejlepší rady, co lze k práci dostat.

I přes velkou snahu nikdy není možné do šablony zahrnout všechny prvky, co budou při tvorbě práce potřeba, a zaručit, že po doplnění textu, obrázků, literatury apod. bude vše v~pořádku pro všechny možné diplomové práce. Bude-li někde delší text, než se předpokládalo, a zalomí-li se na dva řádky, bude-li v literatuře položka, se kterou nebyl otestován využitý styl, a v dalších případech může být výsledek neuspokojivý a může být potřeba do~šablony zasáhnout a chybu, která se projevuje třeba jen pro jednu práci ze sta, opravit. Výsledné PDF a následně i vytištěnou papírovou verzi je tedy vždy nutné pečlivě zkontrolovat a~nespoléhat se na to, že \uv{tohle přece generuje šablona, tak to musí být správně}. Najdete-li v šabloně nějaké chyby nebo budete-li mít návrhy na její vylepěšení, napište prosím na e-mail \texttt{sablona@fit.vutbr.cz} a pomozte nám s jejím vylepšováním. Veškeré připomínky a návrhy jsou vítány.

S kontrolou výsledku může výrazně pomoci vedoucí práce. Nelze však předpokládat, že vedoucí poslední noc před odevzdáním bude sedět v práci připraven na kontrolu desítek stran textu. Proto je nutné mít vše připravené v předstihu a konzultovat průběžně. Kritický pohled vedoucího pak umožní dosažení kvalitního výsledku a aktivita, kterou uvidí, přispěje k pozitivnímu hodnocení práce z jeho strany.

Na závěr bych jménem autorů této šablony popřál všem, kteří právě vytvářejí svoje diplomové práce nebo se k jejich tvorbě připravují, úspěšné dokončení a obhájení práce.




%===============================================================================
