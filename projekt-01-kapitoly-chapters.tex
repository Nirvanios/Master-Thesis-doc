% Tento soubor nahraďte vlastním souborem s obsahem práce.
%=========================================================================
% Autoři: Michal Bidlo, Bohuslav Křena, Jaroslav Dytrych, Petr Veigend a Adam Herout 2019
\chapter{Úvod}
\label{chapter:uvod}

\chapter{Teorie}
Následující kapitola je věnována vysvětlení základní teorie skrývající se za simulacemi a animacemi tekutin a plynů. Nejprve je zde obecně vysvětlen pojem simulace a animace kapalin. Dále jsou zde vysvětleny různé přístupy k simulaci, je zde vysvětlena mřížková neboli Lagrangeova metoda a částicový přístup neboli Eulerova metoda. V každé sekci jsou následně popsány některé významné algoritmy spadající pod jednotlivé metody.

\label{chapter:teorie}
\section{Simulace kapalin}
\section{Eulerova metoda toku}
\subsection{Celulární automaty}
\subsection{Mřížková metoda}
\section{Lagrangeova metoda toku}
\subsection{Smoothed Particle Hydrodynamics}
\subsubsection{WCSPH}
\subsubsection{PCISPH}
\section{Hybridní přístup}




\chapter{Návrh řešení}
\label{chapter:navrh_resení}

\chapter{Implementace}
\label{chapter:implementace}

\chapter{Testování}
\label{chapter:testovani}

\chapter{Závěr}
\label{chapter:zaver}





%===============================================================================
