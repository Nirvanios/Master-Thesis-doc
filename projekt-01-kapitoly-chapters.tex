% Tento soubor nahraďte vlastním souborem s obsahem práce.
%=========================================================================
% Autoři: Michal Bidlo, Bohuslav Křena, Jaroslav Dytrych, Petr Veigend a Adam Herout 2019
\chapter{Úvod}
\label{chapter:uvod}

\chapter{Teorie}
Následující kapitola je věnována vysvětlení základní teorie skrývající se za simulacemi a animacemi tekutin a plynů. Nejprve je zde obecně vysvětlen pojem simulace a animace kapalin. Dále jsou zde vysvětleny různé přístupy k simulaci, je zde vysvětlena mřížková neboli Lagrangeova metoda a částicový přístup neboli Eulerova metoda. V každé sekci jsou následně popsány některé významné algoritmy spadající pod jednotlivé metody.

\label{chapter:teorie}
\section{Simulace kapalin}
Na začátek je nutné odlišit pojmy simulace kapalin a animace kapalin. V obou případech jde především o chování kapaliny v určité situaci, nicméně animace se oproti simulaci zaměřuje především na vizuální stránku a méně na fyzikální přesnost. Jedná se tedy pouze o aproximaci vzorců popisujících chování kapalin například zanedbáním velké části objemu kapaliny a popisem pouze chování hladiny kapaliny (viz níže). Jedním z možných využití takovýchto aproximací jsou případy, kde natolik nezáleží na přesnosti chování, jako spíše na vizuální stránce. Příkladem mohou být videohry, kde je nutné mít vizuálně přívětivou kapalinu, ale zároveň vypočitatelnou v reálném čase s přihlédnutím na výpočet mnoha dalších věcí v rámci herního enginu. Avšak například při tvorbě animací a vizuálních efektů pro filmy, kde nejsme omezeni časem, lze využívat mnohem výpočetně náročnějších animací. Tyto animace sice stále nemusí být fyzikálně dokonalé, avšak se jedná o mnohem propracovanější vyobrazení kapalin, než pouhé vlnění hladiny.

\begin{figure*}[h]\centering
  \centering
  \captionsetup{justification=centering}
  \includegraphics[width=0.3\linewidth]{obrazky-figures/SurfaceWaves/SurfWave_01.png}\hfill
  \includegraphics[width=0.3\linewidth]{obrazky-figures/SurfaceWaves/SurfWave_02.png}\hfill
  \includegraphics[width=0.3\linewidth]{obrazky-figures/SurfaceWaves/SurfWave_03.png}\hfill
  \includegraphics[width=0.3\linewidth]{obrazky-figures/SurfaceWaves/SurfWave_04.png}\hfill
  \includegraphics[width=0.3\linewidth]{obrazky-figures/SurfaceWaves/SurfWave_05.png}\hfill
  \includegraphics[width=0.3\linewidth]{obrazky-figures/SurfaceWaves/SurfWave_06.png}\hfill
  \caption{\textbf{Shallow Water Equation.} Postupné šíření vln při několika kapkách vody.}
  \textbf{Zdroj: } \url{https://en.wikipedia.org/wiki/Shallow_water_equations}
  \label{fig:SWE}
\end{figure*}

\subsubsection{Simulace hladiny}
Simulace vlnění hladiny je jedním z nejjednodušší vizualizace kapaliny. Existuje samozřejmě více přístupů jak takovou simulaci realizovat, mezi které patří například výškové mapy, vlnová funkce či rovnice mělké vody (Shallow Water Equation). Ačkoliv produkují v celku uspokojivé vlnění hladiny, existují v okolním světě běžné jevy, jako například lámající se vlny, které za pomocí jednoduchých vlnových funkcí a výškových map nelze simulovat.

\begin{figure}[hbt]
	\centering
	\captionsetup{justification=centering}
	\includegraphics[width=0.4\textwidth]{obrazky-figures/Large_breaking_wave.jpg}
	\caption{\textbf{Lámající se vlna.} Jev v reálném světě, který není realizovatelný za pomocí výškových map a vlnových funkcí.}
	\textbf{Zdroj: } \url{https://en.wikipedia.org/wiki/Breaking_wave}
	\label{keepCalm}
\end{figure}

Simulace kapalin se naopak snaží být oproti animacím co nejvíce fyzikálně přesné, čímž ale rapidně vzrůstá náročnost výpočtů. Existuje nespočet metod jak dosáhnout výsledku, nicméně čím přesnější a složitější scéna tím delší výpočet daného scénáře. Doba výpočtu se tak může pohybovat v rámci minut, ale i hodin. Tyto metody jsou pak často založeny nad numerickými výpočty fyzikálních rovnic, kde nejpoužívanější rovnice jsou rovnice Navier-Stokes.

\subsection{Navier-Stokes rovnice}
Navier-Stokes rovnice jsou jedny z nejvyužívanějších rovnic pro výpočty chování kapalin. Základy pro popis dynamiky kapalin položil již v roce 1687 Sir Isaac Newton v článku "Principa", kde bylo poprvé správně popsáno chování viskózních kapalin. Později Daniel Bernoulli a Leonhard Euler popsali chování neviskózního toku pomocí rovnic dnes známých jako Eulerovy nevyskózní rovnice (Euler’s inviscid equations). Až Claude-Louis Navier a Sir George Stokes na sobě nezávisle odvodili finální podobu rovnic ze zmiňovaných Eulerových a Newtonových rovnic. Tyto Navier-Stokesovi rovnice jsou nyní nejpoužívanější formou popisu chování kapalin a bylo na nich postaveno nespočet různých algoritmů. \cite{simscale_2020}

\begin{equation}
	\rho(\frac{\partial}{\partial t} + \mathbf{u} \cdot \nabla)\mathbf{u} = -\nabla p + \mu\nabla\cdot(\nabla \mathbf{u}) + f 
	\label{eq:NavierStokes}
\end{equation}

\begin{equation}
	\nabla \cdot \mathbf{u} = 0
	\label{eq:NavierStokes2}
\end{equation}

Tři základní vlastnosti viskózní kapaliny, která má neměnné teplo, jsou rychlost ($\mathbf{u}$), tlak ($p$) a hustota ($\rho$). V rovnici \ref{eq:NavierStokes} pak $\mu$ označuje viskozitu kapaliny a $f$ ostatní síly působící na kapalinu, jako například gravitace. Tyto rovnice tak vyjadřují zákon o zachování hybnosti a zákon o zachování hmotnosti pro Newtonské kapaliny. Newtonosvksá kapalina je pak kapalina kterou lze popsat lineárním Newtonovým modelem. Její viskozita je pak závislá především na tlaku a teplotě a z tohoto hlediska se jedná o takzvanou Newtonovskou viskozitu. u Nenowtonské kapaliny pak popisujeme takzvanou zdánlivou viskozitu závislou na předchozí deformaci kapaliny a rychlosti vnitřního smyku kapaliny. Tyto kapaliny pak nelze popsat lineárním Newtonovým zákonem.\cite{StejskalJan2013Pmks}

Navier-Stokesovi rovnice jsou hojně využívány nejen pro simulaci a animaci kapalin, ale v celé řadě dalších vědních oborů. Využití můžeme nalézt při vytváření modelů pro předpověď počasí, při výrobě letadel pro studování toku vzduchu nebo například při analýze šíření znečištění.
\break

Existují dva úhly pohledu na organizaci a řešení nejen těchto rovnic, ale zároveň obecně dynamiky kapalin. Tyto metody se pak liší především v bodech pozorování. V Eulerově metodě dochází k výpočtu požadovaných hodnot v přesně daných bodech, tedy v nějaké diskrétní mřížce. U Lagrangeovy metody dochází k výpočtu v bodech sledované masy, tedy v částicích sledované kapaliny. Dále jsou obě metody detailněji popsány, včetně několika algoritmů, které pod dané metody spadají.

\begin{figure}[h]
\centering
\begin{subfigure}{.5\textwidth}
  	\centering
	\includegraphics[width=0.7\linewidth]{obrazky-figures/EulerLagran_02.jpg}
	\caption{\textbf{Eulerův přístup}}
	\label{fig:Euler}
\end{subfigure}%
\begin{subfigure}{.5\textwidth}
  	\centering
	\includegraphics[width=0.7\linewidth]{obrazky-figures/EulerLagran_01.jpg}
	\caption{\textbf{Lagrangeův přístup}}
	\label{fig:Lagran}
\end{subfigure}
\caption{Příklad dvou přístupů nad křižovatkou s auty. Eulerův přístup (\ref{fig:Euler}) sleduje křižovatku v předem daných místech (ramena a střed křižovatky). Naopak Lagrangeův přístup(\ref{fig:Lagran}) sleduje konkrétní vozidla (a,c), jak projíždí křižovatkou.}
\textbf{Zdroj:} \url{http://abe-research.illinois.edu/faculty/dickc/Engineering/ELdescrip2a.htm}
\label{fig:ztencovani}
\end{figure}

\section{Eulerova metoda toku}
Jak bylo popsáno výše, při využití Eulerovy metody, jsou vlastnosti kapaliny počítány v pevně definovaných bodech diskrétní mřížky. Ačkoliv některé vlastnosti kapaliny popisuje Eulerovca metodda mnohem přesněji, největší nevýhodou je samotná mřížka. Jeden z prvních problémů je hrubost mřížky. Podíváme-li se na Obr.\ref{fig:EulerGrid} je patrné, že reálná hladina je lehce zvlněná, ale z důvodu velké hrubosti mřížky a tedy hrubějšímu vzorkování, by lehké zvlnění bylo zanedbáno a hladina by byla rovná. 

Dalším problémem, je paměťová náročnost. Snažíme-li se vytvořit jemnější mřížku, abychom odstranili problémy související s hrubou mřížkou, musíme počítat se zvyšující se paměťovou náročností. Pokud uvažujeme 3D prostor a chceme-li zpřesnit v každé ose mřížku 2x, musíme počítat s 8x více buňkami v paměti. Tento problém však naštěstí lze již řešit pomocí různých struktru jakou jsou například řídké mřížky.
\todo{popis řídké mřížky}

Posledním problémem je samotná uzavřenost mřížky, která brání k plně dynamické simulaci. Kapalina je tedy uzavřena do \enquote{nerozbitné nádoby} a jakýkoliv pokus o její tok mimo mřížku je nemožný. I na tento problém však existuje řešení, a to v podobě dynamických mřížek, které se v případě nutnosti dokáží rozšiřivat v prostoru a poskytnout tak prostor pro rozsáhlejší simulační prostor.
\cite{KelagerSPH} 

\begin{figure}[hbt]
	\centering
	\includegraphics[width=0.4\textwidth]{obrazky-figures/GridEuler.PNG}
	\caption{\textbf{Eulerova mřížka.} Kapalina uzavřená ve 2D mřížce. Rychlost kapaliny je reprezentována ve vyznačených tečkách.}
	\textbf{Zdroj: } Lagrangian Fluid Dynamics Using Smoothed Particle Hydrodynamics \cite{KelagerSPH}
	\label{fig:EulerGrid}
\end{figure}

\subsection{Mřížková metoda}
Mřížková (Grid) metoda, je Eulerovská metoda využívající pro pohyb kapaliny několika polí vektorů. Základem mřížkové metody je pole vektorů rychlostí pro každý zkoumaný bod v mřížce. Toto pole pak představuje pohyb tekutiny v celém zkoumaném prostoru. \todo{maybe figure}
\begin{equation}
    \Vec{u}(x,y) = (u_x,u_y)
\end{equation}
Dalším základním stavebním kamenem algoritmu je pak advekce, neboli přesun vlastností z jednoho místa na jiné v důsledku pohybu kapaliny. Uvažujeme-li, že tok kapaliny přenáší například určitou koncetraci částic (kouř, barvivo, písek), pak máme dvě možnosti jak posunout dané hotnoty v čase a prostoru.  První možností je dopředný posun v čase. V závistolsti na pozici $\mathbf{r}$ zvolíme odpovídající rychlost $\mathbf{u}$ a danou hodnotu $A$ posuneme v prostoru. 

\begin{equation}
     A(\mathbf{r} + \mathbf{u}\Delta t, t + \Delta t) = A(\mathbf{r}, t)
\end{equation}

Druhým možným přístupem pro přesun hodnot je tzv. backtraking. V tomto přístupu není hodnota posunuta z jedné pozice na druhou. Nicméně se vektor rychlosti invertuje a do nynější pozice se přesouvá pozice předchozí.

\begin{equation}
    A(\mathbf{r}, t + \Delta t) = A(\mathbf{r} - \mathbf{u}\Delta t, t)
\end{equation}

Stejně jako přesouváme určité částice v prostoru můžeme přesouvat i pole rychlostí a měnit jej tak v čase. V tomto případě však můžou nastat problémy s nestlačitelností a zákone zachování hmotnosti. Musíme tedy zaručit že divergence pole rychlosti je všude nulová. Divergence nám zjednodušené říká, zda v určitém bodě daná vlastnost roste, či klesá. Chceme-li tedy docílit nulové divergence pole rychlosti, chceme vlastně docílit toho, že nám v žádném bodě neklesá ani neroste hustota kapaliny. \cite{webglFluid}

\begin{figure}[h]
\centering
\begin{subfigure}{.3\textwidth}
  	\centering
	\includegraphics[width=0.35\linewidth]{obrazky-figures/div-negative.png}
	\caption{\textbf{Záporná divergence}}
	\label{fig:Euler}
\end{subfigure}%
\begin{subfigure}{.3\textwidth}
  	\centering
	\includegraphics[width=0.35\linewidth]{obrazky-figures/div-positive.png}
	\caption{\textbf{Kladná divergence}}
	\label{fig:Lagran}
\end{subfigure}
\begin{subfigure}{.3\textwidth}
  	\centering
	\includegraphics[width=0.8\linewidth]{obrazky-figures/div-zero.png}
	\caption{\textbf{Nulová divergence}}
	\label{fig:Lagran}
\end{subfigure}
\caption{Příklady různých divergencí v bodě, v závislosti na velikosti a směru vektorového pole v okolí.}
\textbf{Zdroj:} \url{https://www.karlsims.com/fluid-flow.html}
\label{fig:div}
\end{figure}

Při výpočtu pole s nulovou divergencí nám pomůže takzvaný Helmholtz-Hodgeův rozkladový teorém.

\begin{equation}
\mathbf{u} = \mathbf{w} - \nabla p    
\label{eq:HelmHodge}
\end{equation}

Kde $\mathbf{u}$ je naše hledané pole s nulovou divergencí, $\mathbf{w}$ je pole rychlostí s nenulovou divergencí a $\nabla p$ je gradient tlaku. Tato rovnice nám tedy říká, že pole rychlostí, s nenulovou divergencí může být opraveno pomocí gradientu tlaku. Pro výpočet tlaku, lze pak odvodit ze stejného rozkladu \ref{eq:HelmHodge} následující rovnici.

\begin{equation}
    p_{x,y}^{(k+1)} = \frac{p_{x+1,y}^{(k)} + p_{x-1,y}^{(k)} + p_{x,y+1}^{(k)} + p_{x,y-1}^{(k)} + \alpha b_{x,y}}{\beta}
\end{equation}

Kde pro neviskózní kapaliny platí, že $\alpha = -( velikost\_bunky )^2$, $\beta = 4$ a $b = \nabla \cdot \matbf{w}$. Tato rovnice je pak řešena Jacobiho iterativní metodou, kde počáteční odhad je nulový tlak ve všech bodech. Po dostatečném počtu iterací dostaneme hodnotu tlaku ve všech bodech a po výpočtu gradientu a aplikování ve vzorci \ref{eq:HelmHodge} získáme pole rychlostí s nulovou divergencí. Pomocí tohoto pole pak lze posouvat výše zmíněné hodnoty jako je barva, koncentrace částic a jiné.
\cite{GPUGemsGridFLuid}

\begin{figure}[hbt]
	\centering
	\captionsetup{justification=centering}
	\includegraphics[width=0.5\textwidth]{obrazky-figures/GridFluid.jpg}
	\caption{\textbf{Mřížková metoda.} Vizualizace barviva v proudící vodě.}
	\textbf{Zdroj: } Chapter 38. Fast Fluid Dynamics Simulation on the GPU \cite{GPUGemsGridFLuid}
	\label{fig:EulerFluid}
\end{figure}
 
 \todo{využití, výhody/nevýhody}
\subsection{Celulární automaty}
Pro simulaci toku kapalin lze využít i celulárních automatů. Celulární automat se skládá z několika důležitých částí, stavového prostoru rozděleného na diskrétní buňky, přechodové funkce a množiny stavů které mohou buňky nabývat. Celulární automaty pracují na principu $n$-dimenzionálního okolí, kde $n$ závisí na dimenzi zkoumaného prostoru a typu okolí. V jednom časovém kroku se pak aplikuje na všechny buňky přechodová funkce, která vyhodnotí stavy okolních buněk a podle výsledku nastaví vlastní stav. \missingfigure[]{mealy,moore}
Pomocí popsaného principu a využití několika jednoduchých pravidel pak lze vytvořit velice jednoduchý simulátor toku vody. Nejdříve se simuluje působení gravitace a automat se pokusí co nejvíce vody přesunout o buňku níže, pokud je buňka plná, nebo se jedná o buňku, do které nemůže téci voda, zbylá kapalina je distribuována do okolních horizontálních buněk. Jedná se o velice jednoduchý algoritmus, který však zanedbává mnoho vlastností kapalin. Prvním problém je situace při spojených nádobách, pomocí tohoto algoritmu nedojde k vyrovnání hladin a je nutné aplikovat další procesy pro vyrovnávání hladin. Dalším problémem je odrazivost a rychlost toku kapalin. Kapalina se bude vždy pohybovat stejnou rychlostí a představíme li si situaci kdy \enquote{naráží} do stěny, pak nedojde k roztříštění a případnému stoupání kapaliny vzůru. Dalším problémem mohou být vlastnosti kapaliny jako je například viskosita, která se zanedbává, případně může být simulována pouhým koeficientem ovlivňujícím rychlost toku.
\todo{využití, obrázky}

\section{Lagrangeova metoda toku}
Základem Lagrangeovy metody toku není sledování vymezeného prostoru, kde se může kapalina pohybovat, jako je tomu u Eulerovy metody, ale sledovaní kapaliny samotné. Lagrangeova metoda se tak soustředí na kapalinu samotnou, kterou dělí na samostatné části. Můžeme tedy říci, že se jedná o částicovou metodu, kde každá taková částice kapaliny nese specifické parametry jako hustotu, tlak, hmotnost a jiné. Pomocí těchto parametrů pak ovlivňuje ostatní částice v okolí a tím i celou masu kapaliny.

Jak z popisu vyplývá, tyto metody nejsou v prostoru omezeny žádnou mřížkou, či jinou strukturou, limitující prostor dané simulace. A ačkoliv z nutnosti mít mřížku nevzniká vysoká paměťová náročnost, vyvstává nám paměťová náročnost v podobě počtu částic. Chceme-li mít dostatečně přesnou a jemnou simulaci, je nutné mít vysoký počet částic. Počty částic se mohou pohybovat v řádech od desítek tisíc až po jednotky či desítky miliónů. Předpokládáme-li tak vysoký počet částic a skutečnost, že každá částice o sobě musí nést mnoho informací jako je rychlost, pozice, hmotnost a další, čímž se paměťové nároky zvyšují na jiném místě. 
\subsection{Smoothed Particle Hydrodynamics}

\subsubsection{Weakly Compressible Smoothed Particle Hydrodynamics}
\subsubsection{Predictive-Corrective Incompressible Smoothed Particle Hydrodynamics}
\section{Hybridní přístup}




\chapter{Návrh řešení}
\label{chapter:navrh_resení}

\chapter{Implementace}
\label{chapter:implementace}

\chapter{Testování}
\label{chapter:testovani}

\chapter{Závěr}
\label{chapter:zaver}





%===============================================================================
